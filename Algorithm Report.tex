\documentclass{report}
\usepackage{ctex}
\usepackage[utf8]{inputenc} 
\usepackage[T1]{fontenc}    
\usepackage{graphicx,caption,subcaption,animate}       
\usepackage{amsmath}       
\usepackage{amsfonts}     
\usepackage{ragged2e}
\usepackage{array,tabularx,setspace}
\usepackage{url}
\usepackage{hyperref} 

\hypersetup{
    colorlinks=true,
    linkcolor=blue,
    urlcolor=blue,
}
\urlstyle{same}

\usepackage{geometry}
\geometry{top=2cm}

\usepackage{titlesec}
\titleformat{\chapter} [hang] {\bfseries\Large} 
{\thechapter\hspace{20pt}}  {0pt}  {\Large} 
\titlespacing*{\chapter}{0pt}{20pt}{40pt}

\title{程序设计基础大作业算法报告}
\author{自46\; 陈奕帆 \quad 物理41\; 周启轩}
\date{12/12/2024}

\begin{document}
    \maketitle
    \tableofcontents

    \chapter{设计思路}
    \section{概述}
    该项目全部使用c++语言编写,用户可通过在命令行窗口直接使用键盘交互进行游戏
    \section{核心思路}\noindent
    (\uppercase\expandafter{\romannumeral1})界面显示主要思路:\vspace{0.5em}\par
    (1) 对于每个要显示的对象,均定义一个(或一类)结构体,并使用相应的结构体变量(或\\\hspace*{3.6em}数组)存储特定对象在命令行窗口中的坐标、尺寸等位置信息\vspace{0.3em}\par
    (2) 使用"windows.h"库中的SetConsoleCursorPosition函数控制光标位置,在结构体存储\\\hspace*{3.6em}的坐标位置进行对应符号的打印\vspace{0.3em}\par
    (3) 为了实现某些对象的移动,通过定义特定函数,先将对象原先所在位置的符号全部擦\\\hspace*{3.6em}除(设为空格),然后使对象的所有坐标进行整体平移,并进行重新打印\vspace{0.3em}\par
    (4) 对于传送带、空地、以及当前积木,特别定义了对应的char型数组用以存储积木的有\\\hspace*{3.6em}无以及数值:若无积木,则char型数组的值为"X",并在对应的位置直接打印X;若有\\\hspace*{3.6em}积木,则直接在对应位置打印该数组存储的字符串\vspace{0.5em}\\
    (\uppercase\expandafter{\romannumeral2})指令主要思路:\vspace{0.3em}\par
    (1) 首先由用户输入总共的指令数n,并依次按行输入指令;定义一个Instruction结构体\\\hspace*{3.6em}并声明一个结构体数组instruction[1000],用来存储用户输入的每行指令以及操作数\\\hspace*{3.6em}(若该指令有操作数)\vspace{0.3em}\par
    (2) 对指令从1到n进行循环,首先判断由该指令是否有误(包括超出指令集、操作数异\\\hspace*{3.6em}常、指令内部错误等等),若有误,则输出"Error on instruction x",并结束游戏;若指\\\hspace*{3.6em}令正常,则调用该指令对应的函数\vspace{0.3em}\par
    (3) 对于jump和jumpifzero指令,调用相应函数,使用指针修改第二轮循环过程中指标\\\hspace*{3.6em}i的值即可实现指令效果

    \chapter{工程结构}\label{chap-2}
    \section{项目结构}
    项目工作主要在Visual Studio 2022中进行,项目名称为HUMAN RESOURCE MACHINE,各类文件及其功能主要包括:\vspace{0.5em}\\
    (\uppercase\expandafter{\romannumeral1})头文件:\\
    \begin{tabularx}{1\textwidth}{lX}
    (1) base.h  \hspace*{6.8em}& 由于项目体量较小,且为避免潜在的重复定义,全部变量及函数的声明均被写入该文件
    \end{tabularx}\vspace{0.5em}
    (\uppercase\expandafter{\romannumeral2})源文件:\vspace{0.3em}\\
    \renewcommand{\arraystretch}{1.25}
    \begin{tabularx}{1\textwidth}{lX}
    (1) main.cpp & 项目的启动文件,其中只包含关卡选择界面的编写\\
    (2) initialization.cpp & 用于对四个关卡的显示界面进行初始化(包括输入/输出传送带、空地、机器人、当前积木、指令窗口等)\\
    (3) print.cpp & 调用该文件中的函数可以打印或擦除传送带、机器人等对象,还包括一些能使机器人移动的函数\\
    (4) play.cpp & 不同关卡功能实现的主要文件,其中包含play1、play2、play3、play4四个函数,根据用户选择的关卡调用相应函数即可启动该关卡\\
    (5) function.cpp & 编写指令集中所有指令的功能实现\\
    (6) operatestring.cpp \qquad\qquad& 一些附加函数,用以在字符串与数字之间进行转换,内含判断是否发生转换异常的功能\\
    (7) judge.cpp & 用以判断游戏结束后是否成功通关,并显示"Success"或"Fail"
    \end{tabularx}\vspace{0.8em}
    (\uppercase\expandafter{\romannumeral3})资源文件:\par
    (1) in.txt \hspace{6.6em} 用来存储用户已经通过的关卡数
    \section{模块划分}\noindent
    (1) 界面显示模块主要包含在initialization.cpp和print.cpp中\\
    (2) 指令模块主要包含在function.cpp中,指令相应函数包含了对前端(此处指命令\\\hspace*{1.7em}行窗口的显示)和后端两部分的修改,调用时一方面在后端修改各变量的数值,另一方面\\\hspace*{1.7em}同时对前端的显示做出相应更新\\
    (3) 不同模块之间的交互主要通过调用模块内部相应函数进行

    \chapter{选择关卡界面的设计}
    用户初次打开时,将会看到四个代表关卡的方框,其中只有第1关已解锁,2、3、4关的方框中用"X"表示尚未解锁,如图3.1\vspace{-1em}\\
    \begin{figure}[htbp]
        \centering
        \includegraphics[width=0.8\textwidth]{select_level 1.png}
        \caption*{图3.1}
      \end{figure}\par
    当用户成功通过第1关时,将会返回选择关卡界面,此时第2关也已经解锁,如图3.2\vspace{-1em}\\
    \begin{figure}[htbp]
        \centering
        \includegraphics[width=0.8\textwidth]{select_level 2.png}
        \caption*{图3.2}
    \end{figure}\newpage
    如果用户尝试选择尚未解锁的关卡,窗口会进行提示,并让用户重新选择关卡,如图3.3\vspace{-1em}\\
    \begin{figure}[htbp]
        \centering
        \includegraphics[width=0.8\textwidth]{select_level 3.png}
        \caption*{图3.3}
    \end{figure}\par
    如果用户已经顺利通过了所有关卡,则在选择关卡时可以自由选择,如图3.4\vspace{-1em}\\
    \begin{figure}[htbp]
        \centering
        \includegraphics[width=0.8\textwidth]{select_level 5.png}
        \caption*{图3.4}
    \end{figure}



    \chapter{游戏测试}
    \section{测试目标}
    测试游戏主线的运行是否正常、能否良好地应对不合法输入、窗口显示是否存在缺陷等
    \section{游戏主线测试}
    对于第1、2、3关,如果用户输入正确的通关指令,则会在输出传送带上显示正确的输出序列,并且窗口稍后会显示"Success"代表此关卡成功通关
    \begin{figure}[htbp]
        \centering
        \begin{subfigure}[b]{0.45\textwidth}
            \includegraphics[width=\textwidth,height=0.17\textheight]{test1.png}
            \caption*{图4.2.1:第1关通关界面}
        \end{subfigure}
        \hfill
        \begin{subfigure}[b]{0.45\textwidth}
            \includegraphics[width=\textwidth]{test-2.png}
            \caption*{图4.2.2:第2关通关界面}
        \end{subfigure}
    \end{figure}
    \begin{figure}[htbp]
        \centering
        \begin{subfigure}[b]{0.45\textwidth}
            \includegraphics[width=\textwidth,height=0.17\textheight]{test-3.png}
            \caption*{图4.2.3:第3关通关界面}
        \end{subfigure}
        \hfill
        \begin{subfigure}[b]{0.45\textwidth}
            \includegraphics[width=\textwidth]{test-success.png}
            \caption*{图4.2.4:通关后显示"Success"界面}
        \end{subfigure}
    \end{figure}\vspace{1em}\par
    而如果用户的未达成通关条件,或是指令执行时出现错误,则会显示"Fail"或"Error on instruction X",代表游戏失败或指令错误\newpage
    \begin{figure}[t]
        \begin{subfigure}[b]{0.45\textwidth}
            \includegraphics[width=\textwidth,height=0.17\textheight]{test1-fail.png}
            \caption*{图4.2.5:第1关失败界面}
        \end{subfigure}
        \hfill
        \begin{subfigure}[b]{0.45\textwidth}
            \includegraphics[width=\textwidth]{fail.png}
            \caption*{图4.2.6:失败后显示"Fail"界面}
        \end{subfigure}
    \end{figure}\par
    \begin{figure}[htbp]
        \begin{subfigure}[b]{0.45\textwidth}
            \includegraphics[width=\textwidth,height=0.17\textheight]{test1-error -1.png}
            \caption*{图4.2.7:第1关指令错误界面}
        \end{subfigure}
        \hfill
        \begin{subfigure}[b]{0.45\textwidth}
            \includegraphics[width=\textwidth]{test1-error.png}
            \caption*{图4.2.8:检测到指令错误后显示Error界面}
        \end{subfigure}
    \end{figure}\par

    \section{非法输入的应对}
    (1)选择关卡页面\\
    如果用户输入非整数,则提示格式错误;\\
    如果用户输入整数,但不是1、2、3、4中任意一个,则提示不存在此关卡;\\
    以上情况均要求用户重新输入
    \begin{figure}[htbp]
        \begin{subfigure}[b]{0.45\textwidth}
            \includegraphics[width=\textwidth,height=0.175\textheight]{select-illegal-1.png}
            \caption*{图4.3.1:输入非整数}
        \end{subfigure}
        \hfill
        \begin{subfigure}[b]{0.45\textwidth}
            \includegraphics[width=\textwidth]{select-illegal-2.png}
            \caption*{图4.3.2:输入为整数,但不是1/2/3/4}
        \end{subfigure}
    \end{figure}\newpage
    (2)游戏进行中\\
    如果用户输入的指令数非整数,则提示格式错误,并要求重新输入;\\
    如果用户输入的指令是非法的,在"CODE"栏中仍会正常显示该错误指令;该指令以前的指令会正常运行,运行到该指令时会暂停,等待1s后输出"Error on instruction X"
    \begin{figure}[htbp]
        \begin{subfigure}[b]{0.45\textwidth}
            \includegraphics[width=\textwidth]{game-illegal-1.png}
            \caption*{图4.3.3:指令数非整数}
        \end{subfigure}
        \hfill
        \begin{subfigure}[b]{0.45\textwidth}
            \includegraphics[width=\textwidth]{game-illegal-2.png}
            \caption*{图4.3.4:指令非法}
        \end{subfigure}
    \end{figure}\\

    如果用户输入的文件名不存在,则提示格式错误,并要求重新输入;\\
    \begin{figure}[htbp]
        \centering
        \includegraphics[width=\textwidth]{error 11.png}
        \caption*{文件不存在}
    \end{figure}
            

    \chapter{游戏运行说明}
    \section{文件说明}\noindent
    文件压缩包链接:\;\href{https://github.com/Wallfacer-04/HUMAN-RESOURCE-MACHINE/blob/main/HUMAN%20RESOURCE%20MACHINE.zip}{https://github.com/Wallfacer-04/HUMAN-RESOURCE-MACHINE/blob/\\\hspace*{8em}main/HUMAN\%20RESOURCE\%20MACHINE.zip}\vspace{0.5em}\\
    除在\;\hyperref[chap-2]{2\;工程结构}\;中提到的文件外,该压缩包内还有:\\
    (1)\;level1.txt \;,\; level2.txt \;,\; level3.txt \;,\; level4.txt \\
    \hspace*{1em} 分别为第1、2、3、4关的通关指令(包含指令数) \vspace{0.1em}\\
    (2)\;main1.cpp \quad 这是提交到OJ平台上的代码

    \section{代码运行说明}\noindent
    将压缩包内的代码在visual studio中打开后,即可正常运行\\
    若要直接解锁所有关卡,可先将压缩包内in.txt文件内的数值改为5,之后在visual studio中运行该项目\vspace{0.5em}\par
    \textbf{第一步}\par
    在选择关卡页面用键盘输入1-4中的一个数字,并按一次回车;显示“请按任意键继续”后,再按一次回车,即可进入游戏界面
    \begin{figure}[htbp]
        \centering
        \includegraphics[width=0.8\textwidth]{step1.png}
        \caption*{图5.1}
    \end{figure}\newpage
    \textbf{第二步}\par
    选择使用键盘输入/文件读取(输入"k"代表键盘输入,输入"f"代表文件读取)
    \begin{figure}[htbp]
        \centering
        \includegraphics[width=0.8\textwidth,height=0.32\textheight]{图5.2.1.png}
        \caption*{图5.2}
    \end{figure}\par

    \textbf{第三(一)步}\par
    若输入"k",则用键盘输入指令数,并按一次回车
    \begin{figure}[htbp]
        \centering
        \includegraphics[width=0.8\textwidth,height=0.32\textheight]{step2.png}
        \caption*{图5.3(1)}
    \end{figure}\par
    
    \textbf{第四(一)步}\par
    用键盘逐行输入指令(可直接复制cheat-x.txt文件中的指令),再按一次回车,即可进入运行阶段\newpage
    \begin{figure}[htbp]
        \centering
        \includegraphics[width=0.8\textwidth,height=0.35\textheight]{step3.png}
        \caption*{图5.4(1)}
    \end{figure}\par

    \textbf{第三(二)步}\par
    若输入"f",则输入文件名(包含后缀名),并按一次回车
    \begin{figure}[htbp]
        \centering
        \includegraphics[width=0.8\textwidth,height=0.32\textheight]{图5.3.1.png}
        \caption*{图5.3(2)}
    \end{figure}\par

    \textbf{第四步}\par
    等待机器人自动运行,自动运行期间可通过直接按键盘上的数字键1,2,3,4,5键调节机器人运行速度,1为最慢,5为最快(按键后可能需要等待一会才会改变速度)\par
    运行结束后,窗口左下角显示"请按任意键继续,此时按回车键可进入判断是否成功通关的界面\newpage
    \begin{figure}[htbp]
        \centering
        \includegraphics[width=0.8\textwidth,height=0.35\textheight]{step4.png}
        \caption*{图5.4}
    \end{figure}\par

    \textbf{第五步}\par
    查看判定结果:\;若显示"Success"代表成功通关,若显示"Fail"代表通关失败,若除"Fail"外还显示"Error on instruction X",代表第X条指令有误
    \begin{figure}[htbp]
        \centering
        \includegraphics[width=0.45\textwidth]{step5.png}
        \caption*{图5.5}
    \end{figure}\par
    \begin{figure}[htbp]
        \centering
        \begin{subfigure}[b]{0.45\textwidth}
            \includegraphics[width=\textwidth]{step7.png}
            \caption*{图5.6}
        \end{subfigure}
        \begin{subfigure}[b]{0.45\textwidth}
            \includegraphics[width=\textwidth]{step6.png}
            \caption*{图5.7}
        \end{subfigure}
    \end{figure}\par
    显示判定结果后,按一次回车键即可返回选择关卡界面


    \chapter{自由创新关卡}
    \section{概述}
    该关卡的所有积木运载的单位由数字改为单个字符,用户需借助空地将输入传送带上的积木依照顺序放到输出传送带上
    \section{关卡任务}\noindent
    输入序列:\hspace{3em}r\;,\;w\;,\;d\;,\;l\;,\;o\;,\;h\;,\;e\\
    目标输出序列:\hspace{1em}d\;,\;l\;,\;r\;,\;o\;,\;w\;,\;o\;,\;l\;,\;l\;,\;e\;,\;h\\
    可用空地数:\hspace{2em}3\\
    可用指令集:\hspace{2em}inbox\;,\;outbox\;,\;copyto\;,\;copyfrom
    \section{内容展示}\noindent
    一种可行通关代码:\quad 指令数:28 \quad 指令:\href{https://github.com/Wallfacer-04/HUMAN-RESOURCE-MACHINE/blob/main/cheat-4.txt}{cheat-4.txt}\\
    用户完成第四关的关卡任务后,游戏界面如下:
    \begin{figure}[htbp]
        \centering
        \includegraphics[width=0.8\textwidth]{level4-1.png}
        \caption*{图6.1}
    \end{figure}\newpage
    可以发现,输出传送带上的内容从上到下组成一句话:"Hello world"。用户达成这一任务后,命令行窗口不会输出"Success",而是改为"Hello, world!",如图5.2\\
    \begin{figure}[htbp]
        \centering
        \includegraphics[width=0.8\textwidth]{level4-2.png}
        \caption*{图6.2}
    \end{figure}

    

    \chapter{小组分工}
    \section{成员名单}\noindent
    \begin{tabular}{llll}
        自动化系 & 自46 & 陈奕帆 & 2024010987 \\
        物理系 & 物理41 & 周启轩 & 2024011184 
        \end{tabular}
    \section{具体分工}\noindent
    陈奕帆:\;主要负责前端显示部分的开发,包括选择关卡界面、游戏界面、传送带、机器人等的\\\hspace*{4em}显示以及变动\\
    周启轩:\;主要负责后端运算部分的开发,包括指令相应函数、指令的执行等过程\vspace{0.5em}\\
    项目后期的调试、更新等工作由二人共同负责
    

    \chapter{录屏演示共享链接}
    访问清华云盘:\href{https://cloud.tsinghua.edu.cn/f/0f7982a50b644c09b793/}{https://cloud.tsinghua.edu.cn/f/0f7982a50b644c09b793/}


    \chapter{算法报告源码}
    本算法报告全部使用\LaTeX 编写,源码见\href{https://github.com/Wallfacer-04/HUMAN-RESOURCE-MACHINE/blob/main/Algorithm%20Report.tex}{https://github.com/Wallfacer-04/HUMAN-RESOURCE-MACHINE/blob/main/Algorithm\%20Report.tex}
    
\end{document}
